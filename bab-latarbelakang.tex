\chapter{LITERATURE REVIEW}                                       

\section{Pengenalan}

Bab ini membincangkan konsep literasi dalam konteks pendidikan prasekolah serta potensi penggunaan Augmented Reality (AR) sebagai alat inovatif untuk memperkayakan pengalaman pembelajaran huruf dalam kalangan murid prasekolah. Kajian lampau telah menunjukkan bahawa meskipun kaedah pembelajaran tradisional seperti buku teks dan latihan bertulis masih meluas digunakan, ia menghadapi pelbagai cabaran dalam memastikan murid benar-benar menguasai, mengingati dan berinteraksi dengan konsep literasi secara menyeluruh (Gee, 1999; Jamila et al., 2012). Seiring perkembangan teknologi dan permintaan pendidikan abad ke-21, integrasi teknologi seperti AR dalam literasi awal semakin menjadi tumpuan dalam bidang akademik dan amalan pendidikan, memandangkan potensinya untuk menyediakan pendekatan yang lebih interaktif, menyeronokkan, serta efektif dalam pembelajaran huruf dan kemahiran membaca (Azuma, 1997; Bacca et al., 2014; Chen et al., 2020; Radu, 2014).

\section{Teori New Literacy Studies (NLS)}
2.2.1Pengenalan kepada NLS
Dalam era pendidikan moden, literasi tidak lagi terhad kepada kemahiran membaca dan menulis secara mekanikal semata-mata, tetapi telah berkembang menjadi satu amalan sosial dan budaya yang dipengaruhi oleh kemajuan teknologi, interaksi antara individu, serta perkembangan dunia digital (Gee, 1999; Kementerian Pendidikan Malaysia, 2013; Bower et al., 2014). Perubahan ini menuntut murid untuk bukan sahaja menguasai literasi asas, tetapi juga kebolehan menggunakan teknologi dan berinteraksi dalam pelbagai konteks pembelajaran abad ke-21 (UNESCO, 2022)
James Paul Gee (1999) merupakan salah seorang sarjana yang memperkenalkan New Literacy Studies (NLS), yang menekankan bahawa literasi bukan hanya merujuk kepada keupayaan membaca dan menulis secara formal, tetapi juga bagaimana seseorang menggunakan bahasa dan komunikasi dalam situasi sosial yang lebih luas.


Street (2003) pula memperkukuhkan konsep ini dengan menyatakan bahawa literasi bukan sekadar kecekapan kognitif, tetapi turut berkait rapat dengan budaya, masyarakat, dan teknologi. Kajian NLS menunjukkan bahawa literasi berkembang mengikut keperluan sosial, di mana seseorang bukan hanya memahami perkataan, tetapi juga menggunakannya, menyesuaikan diri, serta berinteraksi dengan maklumat dalam konteks dunia sebenar.
\section{Implikasi NLS dalam Pendidikan Moden}
Teknologi Augmented Reality (AR) memainkan peranan penting dalam mengukuhkan pendekatan New Literacy Studies, kerana ia mengubah cara pelajar berinteraksi dengan bahan pembelajaran.
Pembelajaran huruf tidak lagi terbatas kepada format dua dimensi (2D) yang statik, tetapi telah berkembang kepada bentuk tiga dimensi (3D) yang lebih interaktif. Dengan teknologi AR, pelajar dapat melihat, menyentuh, dan mendengar huruf serta perkataan dalam cara yang lebih dinamik, seterusnya membantu mereka memahami konsep dengan lebih mendalam.
Tambahan pula, AR membolehkan murid memahami bukan sahaja bentuk huruf, tetapi juga cara huruf digunakan dalam kehidupan sebenar melalui visualisasi dan animasi. Pendekatan ini selari dengan konsep NLS, yang menekankan bahawa literasi moden bukan sekadar kemahiran teknikal, tetapi turut diperkaya dengan konteks sosial dan teknologi.
\section{Kajian Terdahulu Mengenai NLS dan AR dalam Pendidikan}
Beberapa kajian terdahulu telah membuktikan hubungan antara New Literacy Studies dan penggunaan AR dalam pendidikan, seperti yang ditunjukkan dalam Jadual 2.1:
Jadual  2-1 Hubungan antara New Literacy Studies dan penggunaanAR dalam pendidikan


\documentclass{article}
\usepackage{longtable}
\usepackage{lscape}
\usepackage[margin=2.5cm]{geometry}
\renewcommand{\arraystretch}{1.2}
\usepackage{array}

\begin{document}
\begin{landscape}

\begin{longtable}{|p{4.5cm}|p{4.5cm}|p{5cm}|p{4cm}|p{6cm}|}
\caption{Gabungan Kajian AR dalam Pendidikan} \label{tab:kajian_ar} \\
\hline
\textbf{Pengkaji dan Tahun} & \textbf{Fokus Kajian} & \textbf{Kaedah Kajian} & \textbf{Saiz Sampel / Bilangan Kajian} & \textbf{Dapatan Utama} \\
\hline
\endfirsthead
\hline
\textbf{Pengkaji dan Tahun} & \textbf{Fokus Kajian} & \textbf{Kaedah Kajian} & \textbf{Saiz Sampel / Bilangan Kajian} & \textbf{Dapatan Utama} \\
\hline
\endhead
Billinghurst \& Duenser (2012) & Kesan AR dalam pendidikan & Kajian kes buku AR dan aplikasi mudah alih & Tidak dinyatakan & AR meningkatkan daya ingatan dan kefahaman \\
\hline
Luo et al. (2018) & Alat AR untuk kejuruteraan pembinaan & Kajian kes dengan soal selidik dan temu bual & 40 pelajar & Hasil pembelajaran dan visualisasi lebih baik berbanding kaedah 2D \\
\hline
Kaur et al. (2020) & AR untuk pembelajaran interaktif & Intervensi bilik darjah dan soal selidik model ARCS & 34 pelajar & AR tingkatkan motivasi dan kaedah eksplorasi konsep \\
\hline
Tuli et al. (2020) & AR dalam pendidikan sains dan kejuruteraan & Ujian pra-pasca dan analisis prestasi pelajar & Pelajar sekolah menengah dan kejuruteraan & AR bantu pemahaman elektromagnet dan sebagai makmal maya \\
\hline
Jesionkowska et al. (2020) & AR dalam pendidikan STEAM & Pembelajaran berasaskan projek dengan Unity dan HoloLens & 19 peserta (guru dan pelajar) & AR tingkatkan motivasi dan pembangunan kemahiran \\
\hline
Fitria et al. (2023) & Simulasi AR dalam kejuruteraan mekanikal & Kuasi-eksperimen dengan ujian pra dan pasca & 70 pelajar & AR bantu kefahaman dan kemahiran amali pelajar \\
\hline
\end{longtable}

\end{landscape}
\end{document}




2.4.1Implikasi kepada Kajian Ini
Kajian ini menerapkan prinsip New Literacy Studies (NLS) melalui pembangunan aplikasi AR Alphabets, bagi menilai bagaimana teknologi ini dapat membantu meningkatkan literasi awal murid prasekolah secara interaktif dan visual.
Teknologi Augmented Reality (AR) berfungsi sebagai pelengkap kepada pendekatan NLS, membolehkan murid berinteraksi dengan huruf dan bunyi dalam bentuk yang lebih menarik dan berkesan. Kajian ini akan membandingkan keberkesanan AR dengan kaedah pembelajaran tradisional, sejajar dengan pandangan NLS yang menekankan bahawa literasi moden perlu berkembang selaras dengan perubahan teknologi.
2.4.2Kesimpulan
Bab ini telah mengembangkan Teori New Literacy Studies (NLS) secara lebih mendalam dan kritikal, menjelaskan bahawa literasi bukan sekadar kemahiran membaca dan menulis, tetapi juga praktik sosial yang diperkaya dengan teknologi seperti AR.
Kajian ini akan meneliti bagaimana integrasi AR dapat memperkukuhkan pembelajaran literasi awal, memberikan pengalaman pembelajaran yang lebih dinamik, interaktif, dan efektif bagi murid prasekolah.
Multiliteracies dan Literasi Digital
Dalam dunia pendidikan moden, konsep literasi tidak lagi terbatas kepada bacaan dan penulisan dalam bentuk teks sahaja, tetapi telah berkembang kepada pelbagai bentuk komunikasi yang lebih luas dan kompleks.
Kalantzis dan Cope (2000) memperkenalkan konsep Multiliteracies, yang memberi tumpuan kepada cara manusia berkomunikasi melalui pelbagai saluran, termasuk visual, auditori, digital, dan multimodal. Konsep ini berkembang selaras dengan perubahan dalam cara maklumat disampaikan dan diterima oleh masyarakat moden.
Dengan kepesatan teknologi dan globalisasi, pembelajaran tidak lagi tertumpu kepada buku teks dan tulisan sahaja, tetapi merangkumi pelbagai medium komunikasi digital, seperti grafik, video, animasi, dan interaksi teknologi.
Pendekatan Multiliteracies menekankan bahawa pelajar tidak hanya berinteraksi dengan teks bertulis, tetapi juga dengan imej, bunyi, animasi, serta teknologi digital, yang menjadi sebahagian daripada pengalaman pembelajaran mereka. Dalam dunia yang dipenuhi dengan maklumat digital, pelajar perlu menguasai bukan sahaja literasi tradisional, tetapi juga kemahiran teknologi untuk memahami dan memproses maklumat dengan lebih efektif dan efisien.
2.5.1Multiliteracies dan Literasi Digital dalam Konteks Pendidikan
Literasi digital merujuk kepada keupayaan seseorang dalam memahami, menilai, dan menggunakan teknologi untuk mengakses dan menganalisis maklumat (UNESCO, 2022).
Dalam sistem pendidikan yang semakin bergantung kepada teknologi, pelajar perlu menguasai kemahiran digital bagi memahami dunia yang semakin pantas dan interaktif. Kajian menunjukkan bahawa pelajar yang mempunyai literasi digital yang tinggi lebih cenderung untuk memahami konsep pembelajaran dengan lebih mendalam dan mampu menyesuaikan diri dengan pelbagai bentuk komunikasi (Buckingham, 2008).
Teknologi seperti Augmented Reality (AR) dan Virtual Reality (VR) kini digunakan untuk memperkukuhkan literasi digital, dengan membolehkan pelajar berinteraksi dengan konsep pembelajaran dalam persekitaran maya yang lebih realistik (Billinghurst & Dünser, 2012).
Sebagai sebahagian daripada perkembangan teknologi pendidikan, literasi digital bukan sahaja penting untuk mencapai kefahaman akademik, tetapi juga bagi memperluaskan keupayaan pelajar dalam berkomunikasi dan menyelesaikan masalah dalam dunia sebenar.
2.5.2Augmented Reality (AR) sebagai Alat Pembelajaran Multimodal
Teknologi Augmented Reality (AR) memainkan peranan penting dalam perkembangan konsep Multiliteracies, kerana ia membolehkan pelajar berinteraksi dengan bahan pembelajaran melalui pelbagai saluran komunikasi digital.
AR dikategorikan sebagai alat pembelajaran multimodal, kerana ia melibatkan visualisasi 3D, animasi, bunyi, dan interaktiviti, sejajar dengan pendekatan multiliterasi dalam pendidikan moden.
AR membolehkan pelajar melihat objek maya dalam persekitaran fizikal, yang memberikan mereka pengalaman pembelajaran yang lebih immersif dan realistik. Dalam konteks literasi awal, AR membantu murid mengenali huruf bukan hanya sebagai simbol statik, tetapi sebagai elemen yang hidup, boleh disentuh, serta didengar.
Kajian menunjukkan bahawa pelajar yang menggunakan AR dalam pembelajaran cenderung untuk lebih cepat memahami konsep, berbanding mereka yang menggunakan kaedah tradisional (Billinghurst & Dünser, 2012).
2.5.3Kajian Terdahulu Mengenai Multiliteracies dan Literasi Digital
Berikut adalah beberapa kajian terdahulu yang menyokong perkembangan konsep Multiliteracies dan Literasi Digital dalam pendidikan, seperti yang ditunjukkan dalam Jadual 2.2:
 Jadual  2-2 Hubungan antara Multiliteracies dan Literasi Digital dalam Pendidikan
Kajian	Fokus Kajian	Hasil Kajian
Kalantzis & Cope (2000)	Konsep Multiliteracies	Literasi moden merangkumi visual, auditori, dan komunikasi digital.
Buckingham (2008)	Literasi Digital	Pelajar dengan kemahiran literasi digital memahami maklumat dengan lebih baik.
Billinghurst & Dünser (2012)	AR dalam pembelajaran	AR meningkatkan pemahaman konsep melalui pembelajaran multimodal.
Wu et al. (2013)	Interaksi AR dalam pendidikan	Pelajar lebih berkesan memahami konsep pembelajaran apabila menggunakan AR.

2.5.4Implikasi kepada Kajian Ini
Kajian ini akan mengaplikasikan konsep Multiliteracies dan Literasi Digital dalam konteks penggunaan AR Alphabets, untuk melihat bagaimana teknologi AR membantu meningkatkan pemahaman huruf bagi murid prasekolah. AR sebagai alat literasi multimodal, membolehkan murid berinteraksi dengan bahan pembelajaran dalam bentuk 3D, animasi, dan bunyi.  Kajian ini akan membandingkan keberkesanan AR dengan kaedah pembelajaran tradisional, sejajar dengan pendekatan Multiliteracies yang menekankan kepelbagaian komunikasi dalam pembelajaran.  Pelaksanaan aplikasi AR dalam literasi digital boleh digunakan sebagai model dalam pembangunan teknologi pendidikan yang lebih inovatif.
2.5.5Kesimpulan
\section{Bahagian ini telah memperluaskan konsep Multiliteracies dan Literasi Digital, serta menghubungkannya dengan penggunaan Augmented Reality (AR) dalam pendidikan moden. Perbincangan literatur menunjukkan bahawa AR adalah alat pembelajaran multimodal yang mampu meningkatkan pemahaman pelajar secara lebih interaktif dan imersif
Literasi Digital dan Teknologi Pendidikan}
2.6.1Pngenalan kepada Literasi Digita
Dalam era teknologi moden, literasi tidak lagi terbatas kepada keupayaan membaca dan menulis secara konvensional, tetapi telah berkembang kepada keupayaan mengakses, menilai, dan memahami maklumat digital secara kritikal. Literasi digital menjadi semakin penting kerana dunia pendidikan dan industri kini bergantung kepada teknologi sebagai medium komunikasi dan pembelajaran utama (UNESCO, 2022).Menurut Buckingham (2008), literasi digital merangkumi kemahiran mencari, menilai, dan menggunakan maklumat yang diperoleh melalui teknologi. Ini bermakna pelajar bukan sahaja perlu tahu membaca teks tetapi juga memahami kandungan multimedia, menilai kesahihan maklumat, dan menggunakannya secara efektif dalam kehidupan seharian.
2.6.2Kepentingan Literasi Digital dalam Kurikulum Pendidikan
UNESCO (2022) menegaskan bahawa literasi digital perlu menjadi asas utama dalam kurikulum pendidikan kerana ia membantu pelajar berinteraksi dengan maklumat secara aktif dan bermakna.  Kementerian Pendidikan Malaysia (KPM) turut menekankan kepentingan penggunaan teknologi dalam sistem pendidikan, sejajar dengan Pelan Pembangunan Pendidikan Malaysia (PPPM 2013-2025). Dalam pendidikan prasekolah, literasi digital membantu murid beradaptasi dengan teknologi sejak usia muda, membolehkan mereka mengembangkan daya fikir yang lebih kreatif serta memahami maklumat secara visual dan interaktif.
Kajian menunjukkan bahawa pelajar yang mempunyai literasi digital yang tinggi lebih berupaya memahami konsep pembelajaran dengan mendalam dan mampu menyesuaikan diri dengan persekitaran teknologi yang berubah dengan pantas (Kalantzis & Cope, 2000).
2.6.3Kajian Terdahulu Mengenai Literasi Digital dan AR dalam Pendidikan
Berikut adalah beberapa kajian yang menyokong perkembangan literasi digital dan penggunaan AR dalam pendidikan:
Jadual  2-3Literasi Digital dan AR dalam Pendidikan
Kajian	Fokus Kajian	Hasil Kajian
Buckingham (2008)	Literasi Digital	Literasi digital membantu pelajar memahami maklumat secara kritikal.
Billinghurst & Dünser (2012)	AR dalam pembelajaran	AR meningkatkan pemahaman konsep melalui interaksi visual dan auditori.
Wu et al. (2013)	Kesan AR terhadap literasi	AR meningkatkan motivasi dan daya ingatan pelajar.
UNESCO (2022)	Literasi Digital Global	Literasi digital perlu menjadi asas utama dalam pendidikan abad ke-21.
KPM (PPPM 2013-2025)	AR dalam pendidikan Malaysia	KPM menggalakkan penggunaan AR dalam pendidikan untuk meningkatkan keberkesanan pembelajaran.
2.6.4Implikasi kepada Kajian Ini
 Kajian ini akan menguji keberkesanan aplikasi AR Alphabets dalam meningkatkan literasi awal murid prasekolah.  Pelaksanaan AR sebagai alat bantu pembelajaran multimodal akan dinilai melalui pengujian System Usability Scale (SUS) dan kajian keberkesanan interaktif. Kajian ini juga akan melihat bagaimana penggunaan teknologi digital membantu murid memahami dan mengingati huruf dengan lebih cepat dan berkesan
2.6.5Kesimpulan
Bahagian ini telah memperluaskan konsep Literasi Digital dan Teknologi Pendidikan, serta menghubungkannya dengan penggunaan Augmented Reality (AR) dalam pendidikan moden. Perbincangan literatur menunjukkan bahawa literasi digital adalah asas utama dalam pembelajaran abad ke-21, dan AR merupakan salah satu teknologi yang berpotensi memperkaya pengalaman pembelajaran secara interaktif.
\section{ Literasi Berkaitan Augmented Reality (AR)}
Augmented Reality (AR) merupakan teknologi yang menggabungkan elemen digital ke dalam dunia nyata, membolehkan pengguna berinteraksi dengan objek maya dalam persekitaran fizikal (Azuma, 1997). Dalam konteks pendidikan, AR digunakan untuk memperkaya pengalaman pembelajaran, menjadikan konsep abstrak lebih mudah difahami melalui visualisasi interaktif (Billinghurst & Dünser, 2012).Kajian menunjukkan bahawa AR dapat meningkatkan pemahaman pelajar, motivasi pembelajaran, dan daya ingatan, serta membolehkan pelajar mengalami konsep pembelajaran dengan lebih realistik (Wu et al., 2013).
\section{Sejarah Perkembangan Augmented Reality (AR)}
Teknologi AR telah berkembang sejak beberapa dekad lalu, bermula dengan konsep asas sehingga aplikasi moden dalam pelbagai bidang. Berikut adalah perkembangan utama AR:
Jadual  2-4 Sejarah Perkembangan AR
Tahun	Peristiwa Penting	Penerangan
1968	Sensorama dan Head-Mounted Display (HMD) oleh Ivan Sutherland	Ivan Sutherland memperkenalkan paparan HMD pertama yang menjadi asas kepada teknologi AR.
1990	Istilah "Augmented Reality" diperkenalkan	Tom Caudell mencipta istilah Augmented Reality untuk merujuk kepada teknologi yang menggabungkan objek digital dengan dunia nyata.
1997	Kajian AR dalam pendidikan	Ronald Azuma menerbitkan kajian penting mengenai AR, membincangkan keupayaan teknologi ini dalam pelbagai aplikasi, termasuk pendidikan.
2013	Pengenalan Google Glass	Google memperkenalkan Google Glass, peranti AR yang membolehkan maklumat digital dipaparkan dalam bidang penglihatan pengguna.
2016	Pelancaran Pokémon GO	Permainan mudah alih AR pertama yang mencapai kejayaan besar, memperlihatkan potensi teknologi AR dalam industri hiburan dan interaksi pengguna.
2020	AR dalam pendidikan semakin berkembang	AR digunakan secara meluas dalam pembelajaran interaktif, khususnya dalam pembelajaran STEM dan literasi awal kanak-kanak.
	

\section{1Augmented Reality dalam Pendidikan}
Penggunaan AR dalam pendidikan semakin berkembang, dengan pelbagai aplikasi yang membantu pelajar memahami konsep dengan lebih baik. Berikut adalah beberapa manfaat utama AR dalam pendidikan:Meningkatkan pemahaman konsep abstrak melalui visualisasi 3D. Menggalakkan pembelajaran kendiri dengan akses kepada kandungan digital yang lebih menarik. Memudahkan guru menyampaikan konsep yang lebih sukar dengan lebih jelas.  Meningkatkan motivasi dan daya ingatan pelajar melalui interaksi langsung dengan bahan pembelajaran. Kajian menunjukkan bahawa AR boleh digunakan dalam pelbagai bidang pendidikan, termasuk sains, matematik, bahasa, dan Sejarah
2.8.2Kajian Terdahulu mengenai AR dalam Pendidikan
Berikut adalah beberapa kajian terdahulu yang membincangkan kesan penggunaan AR dalam pendidikan:
  Jadual  2-5 Kajian Terdahulu Mengenai AR dalam Pendidikan
Kajian	Fokus Kajian	Hasil Kajian
Sehkar Fayda-Kinik (2023)	Kajian trend AR dalam pendidikan	AR meningkatkan pemahaman dan motivasi pelajar.
Yiannis Koumpouros (2024)	Potensi AR dalam pendidikan	AR membantu pelajar memahami konsep dengan lebih mendalam.
Lampropoulos et al. (2022)	AR dan gamifikasi dalam pendidikan	AR meningkatkan penglibatan dan prestasi akademik pelajar.

2.8.3Kajian Lepas lepas berkaitan penggunaan teknologi Augmented Reality (AR)
Jadual 2.6  menunjukkan ringkasan sepuluh kajian lepas berkaitan penggunaan teknologi Augmented Reality (AR) dalam pendidikan sepanjang lima tahun terkini, khususnya dari segi impak terhadap motivasi, penumpuan, dan pencapaian pelajar.
Jadual  2-6 kajian lepas berkaitan penggunaan teknologi Augmented Reality (AR)
Bil	Tahun	Penulis	Kaedah Kajian	Fokus Kajian	Dapatan
1	2023	Gunalan et al.	Kajian eksperimen	Penggunaan AR dalam pembelajaran sains sekolah rendah	Meningkatkan motivasi dan tumpuan murid melalui visual interaktif dan aktiviti permainan.
2	2022	Lin et al.	Eksperimen (kuasi-eksperimen)	Penggunaan AR dalam pembelajaran matematik sekolah rendah	Meningkatkan penumpuan dan pencapaian murid berbanding kaedah tradisional.
3	2021	Gunawan et al.	Eksperimen	Penggunaan AR dalam pembelajaran sains sekolah menengah	Meningkatkan motivasi intrinsik dan pencapaian ujian pelajar.
4	2020	Sánchez et al.	Eksperimen	Penggunaan AR dalam pembelajaran bahasa asing (Bahasa Inggeris)	Meningkatkan tumpuan dan ingatan pelajar.
Bil	Tahun	Penulis	Kaedah Kajian	Fokus Kajian	Dapatan
5	2019	Norazlina et al.	Kajian kuasi-eksperimen	AR dalam pendidikan prasekolah Malaysia	Peningkatan motivasi dan pencapaian murid dalam pengenalan huruf.
6	2023	Cao & Yu	Meta-analisis	Analisis AR dalam pelbagai disiplin (2016–2023)	Sikap & pencapaian lebih tinggi, tiada perbezaan signifikan motivasi.
7	2025	Ruijia et al.	Kajian sistematik	Motivasi pelajar K-12 dengan AR	Meningkatkan motivasi melalui pengalaman pembelajaran imersif.
8	2023	Özeren & Top	Eksperimen	Penggunaan AR di sekolah menengah	Meningkatkan pencapaian akademik & motivasi berbanding tradisional.
9	2022	Goharinejad et al.	Eksperimen	AR dalam pendidikan pelajar keperluan khas	Mengurangkan defisit perhatian & meningkatkan pembelajaran.
10	2023	Vidak et al.	Kajian sistematik	AR dalam pengajaran fizik	Membantu visualisasi konsep abstrak & meningkatkan pencapaian pelajar.

2.8.4Aplikasi AR
Jadual 2.7  menunjukkan aplikasi menggunakan teknologi Augmented Reality (AR) dalam pendidikan 
Jadual  2-7  Contoh Aplikasi AR
Aplikasi AR	Logo	Huraian
Dinasor3DAR		Aplikasi ini dikhususkan kepada penggemar dinosaur. Pengguna boleh mengarahkan peranti mereka ke langit untuk menyaksikan pengalaman maya yang dipenuhi dengan bintang-bintang. Aplikasi ini amat sesuai untuk kanak-kanak yang teruja menerokai dunia prasejarah dan mempelajari tentang makhluk purba seperti dinosaur.
Earth Zoo AR		Aplikasi interaktif ini membolehkan pengguna menyaksikan muka surat berwarna yang menampilkan objek dalam bentuk 3D. EarthZoo-AR menawarkan pengalaman seolah-olah melawat zoo maya, di mana pengguna dapat menikmati paparan haiwan kegemaran mereka dari sudut pandang yang baharu.
Kad Flash AR		Aplikasi ini bertujuan untuk mendidik kanak-kanak melalui kad flash AR. Dengan pemanfaatan teknologi ini, kanak-kanak dapat melihat huruf dan haiwan dalam bentuk 3D sambil mendengar nama serta bunyi haiwan tersebut, menjadikan proses pembelajaran lebih menyeronokkan dan imersif.
Math ARi		Aplikasi ini direka untuk memperkenalkan kanak-kanak kepada asas-asas pembelajaran seperti huruf, nama haiwan, dan kata-kata melalui interaksi AR yang menghiburkan. Ia amat sesuai untuk kanak-kanak kecil kerana pendekatannya yang mudah dan berkesan.
AR Paint		Aplikasi kreatif ini membolehkan pengguna menghasilkan hologram 3D, lukisan, dan patung. Ia menawarkan peluang untuk meneroka kreativiti, menyertai komuniti seni maya, dan berkongsi hasil karya dengan rakan-rakan melalui teknologi AR.


2.8.5Cara Kerja Augmented Reality AR
Jadual 2.8 menunjukkancara kerja  menggunakan teknologi Augmented Reality (AR) 
Jadual  2-8 Cara  Kerja AR
Software	Description
	Unity 3D adalah perisian untuk mencipta permainan tiga dimensi yang 
digabungkan untuk menghasilkan animasi tiga dimensi secara masa nyata 
(waktu nyata). Unity dilengkapi dengan Persekitaran Pembangunan Terpadu (IDE) 
dikenali sebagai Mono Develop, yang bertujuan untuk mengintegrasikan semua skrip 
dihasilkan ke dalam Unity, untuk diproses secara langsung. Unity 3D 
dibangunkan oleh Unity Technologies, ditubuhkan pada tahun 2004 oleh David 
Helgason, Nicholas Francis, dan Joachim Ante. Pada tahun 2009, Unity dilancarkan 
secara percuma, dan kini ia telah menarik berjuta-juta pembangun dari seluruh dunia 
untuk mendaftar (Rahmat & Yanti, 2021). Unity menyokong pembangunan aplikasi 
Android. Sebelum aplikasi yang dibina menggunakan Unity untuk Android 
boleh dijalankan, konfigurasi persekitaran pembangunan Android pada peranti adalah  diperlukan. Untuk itu, pembangun perlu memuat turun dan memasang Android SDK.  dan menambah peranti fizikal ke dalam sistem. Unity Android membenarkan memanggil fungsi khas yang ditulis dalam C/C++ secara langsung dan Java secara tidak langsung secara tidak langsung melalui skrip C# (Andriansyah et al., 2019).

	Blender merupakan salah satu perisian percuma yang sering dikenali sebagai suite penciptaan 3D sumber terbuka, yang menyokong keseluruhan proses dalam mod tiga dimensi seperti pemodelan, pemasangan rangka, animasi, simulasi, rendering, dan penjejakan gerakan. Malah, perisian ini juga menyokong pembangunan permainan (Valentino, 2017). Pada mulanya, Blender diciptakan sebagai alat pengeluaran dalaman untuk syarikat animasi Belanda yang terkemuka, NeoGeo, yang diasaskan oleh pemaju asal Blender dan masih menjadi pemaju utama hingga ke hari ini, Ton Roosendaal. Menjelang akhir tahun 1990-an, NeoGeo mula menawarkan salinan Blender untuk dimuat turun melalui laman web mereka. Secara perlahan tetapi konsisten, minat terhadap program yang kurang daripada 2 MB ini semakin berkembang. Pada tahun 1998, Ton menubuhkan syarikat baharu, Not a Number (NaN), untuk memasarkan dan menjual Blender sebagai produk perisian. NaN terus menyediakan versi percuma Blender tetapi juga menawarkan versi premium dengan lebih banyak ciri pada harga yang berpatutan. Strategi ini terbukti berkesan, dan pada akhir tahun 2000, pengguna Blender telah mencapai lebih daripada 250,000 di seluruh dunia (Gumster, 2015, hal. 11).
	Vuforia adalah Kit Pembangunan Perisian (SDK) yang dibangunkan oleh Qualcomm untuk membantu pembangun aplikasi mudah alih dalam mencipta aplikasi Realiti Terimbuh (AR) pada telefon pintar sama ada berasaskan Android atau iOS (Rahmat & Yanti, 2021). SDK Vuforia juga boleh diintegrasikan dengan Unity, dikenali sebagai Vuforia AR Extension for Unity. Vuforia AR Extension membolehkan Unity memaparkan animasi realiti terimbuh yang telah direka sebelumnya (Desierto et al., 2020). Untuk berfungsi dengan optimum, SDK Vuforia memerlukan beberapa komponen penting. Komponen tersebut merangkumi kamera, penukar imej, alat pengesan, rendering latar belakang video, kod aplikasi, trackables, dan marker. Semua komponen ini digunakan dalam pembangunan aplikasi berasaskan realiti terimbuh (Mustaqim, 2017).

	Figma merupakan salah satu alat reka bentuk yang sering digunakan untuk mencipta antaramuka aplikasi mudah alih, desktop, laman web, dan banyak lagi. Figma boleh diakses pada sistem operasi Windows, Linux, atau Mac dengan sambungan internet. Keunggulan Figma terletak pada kemampuannya untuk membolehkan lebih dari satu individu berkolaborasi secara serentak, walaupun berada di lokasi yang berbeza. Fenomena ini dapat dianggap sebagai kerja berkumpulan, dan disebabkan oleh keupayaan aplikasi Figma, ia telah menjadi pilihan utama bagi banyak pereka UI/UX untuk menghasilkan prototaip laman web atau aplikasi dengan cara yang cepat dan berkesan (Al-Faruq et al., 2022).
	Adobe Illustrator merupakan aplikasi yang digunakan untuk menyunting reka bentuk grafik dalam penerbitan web dan desktop, serta mampu berintegrasi dengan perisian lain yang relevan (Rian et al., 2021). Adobe Illustrator dapat digunakan untuk menyunting atau mencipta imej dan butang dalam proses pembangunan aplikasi.
,	Teknologi Augmented Reality (AR) telah digunakan secara meluas dalam pelbagai aspek kehidupan, termasuk dalam bidang pendidikan. Pelbagai kajian telah dijalankan untuk menyelidiki bagaimana teknologi ini dapat diterapkan dalam proses pembelajaran pelajar di sekolah. Dalam pangkalan data Google Scholar, pencarian dengan kata kunci "Augmented Reality in Education" pada 25 September 2018 menghasilkan 436,000 hasil pencarian dalam 0.03 saat, dan pada 29 Disember 2019, angka ini meningkat kepada 679,000 hasil dalam 0.07 saat. Pada tahun 2023, hasil pencarian mencapai sekitar 1,340,000 dalam hanya 0.05 saat (Ismayani, 2020). AR dapat digunakan untuk menyediakan pemahaman visual bagi bahan pembelajaran yang sukar untuk dijelaskan hanya dengan tulisan. Sebagai contoh, dalam pengajaran matematik, AR dapat menyertakan animasi yang memperlihatkan perubahan bentuk grafik dan persamaan fungsinya apabila terdapat perubahan pada pembolehubah. AR juga mampu menunjukkan bentuk tiga dimensi bagi objek ruang dan bahagian potongannya yang sebelumnya hanya ditampilkan dalam gambar dua dimensi (Ismayani, 2020). Penggunaan AR membolehkan pengguna bergerak dan mengamati model tiga dimensi yang dipaparkan dari pelbagai sudut. Hal ini menjadikan pembelajaran melalui AR lebih terikat dengan bahan yang dipelajari, dan pengalaman pembelajaran seperti ini menghasilkan proses pembelajaran yang lebih berkesan dan mudah diingati oleh pelajar.
	Android merupakan sistem operasi untuk peranti mudah alih berasaskan Linux yang merangkumi sistem operasi, middleware, dan aplikasi. Android pada asalnya dibangunkan oleh Android Inc., tetapi syarikat ini kemudiannya diambil alih oleh Google pada tahun 2005 (Juhara, 2016). Android bukan sekadar sistem operasi untuk telefon pintar, tetapi juga merupakan pesaing utama Apple dalam segmen sistem operasi tablet. Pertumbuhan pesat Android disebabkan oleh platformnya yang sangat komprehensif, termasuk sistem operasi, aplikasi, alat pembangunan, pasaran aplikasi Android, dan sokongan yang amat tinggi daripada komuniti sumber terbuka global, menjadikan Android terus berkembang. Perkembangannya yang pesat dapat dilihat dari segi teknologi mahupun jumlah peranti di seluruh dunia (Safaat H, 2015). Pengembangan aplikasi Android menawarkan pelbagai pilihan dalam mencipta aplikasi berasaskan Android. Kebanyakan pembangun menggunakan Eclipse, yang tersedia secara percuma dan bebas untuk merancang serta membangunkan aplikasi Android (Safaat H, 2015).

\section{Potensi Augmented Reality dalam Pendidikan}
2.9.1Masa Depan Pendidikan Digital 
AR dijangka menjadi alat utama dalam pendidikan masa depan, membantu pelajar belajar secara lebih mendalam melalui pengalaman langsung.
2.9.2 Integrasi dalam Kurikulum
 AR boleh digunakan untuk memperkaya kurikulum pendidikan, menjadikan pembelajaran lebih interaktif dan menarik.
2.9.3 Peningkatan Teknologi AR 
 Dengan perkembangan teknologi seperti 5G dan peranti pintar, AR akan menjadi lebih mudah diakses dan digunakan dalam bilik darjah.
2.9.4Kesimpulan
Bab ini telah membincangkan literasi berkaitan AR, sejarah perkembangan teknologi ini, penggunaan AR dalam pendidikan, kajian terdahulu dalam bentuk jadual, serta potensi AR dalam pendidikan. Kajian ini bertujuan untuk menilai keberkesanan aplikasi AR Alphabets dalam membantu murid prasekolah mengenali huruf dengan lebih mudah dan menyeronokkan.
\section{Perbandingan Sistem Alfabet dan Penggunaannya}
Jadual 2-1 Perbandingan Sistem Alfabet dan Penggunaannya
Kategori	Bahasa Melayu (Rumi)	Tulisan Jawi	Sistem Alfabet Lain
Jumlah Huruf	26 huruf (A-Z)	37 huruf	Berbeza mengikut bahasa (Contoh: 33 huruf dalam Rusia)
Huruf Unik	Q, V, X jarang digunakan	Tidak mempunyai huruf Latin seperti Q, V, X	Ada huruf unik seperti ñ (Sepanyol) atau ø (Norway)
Penggunaan	Dominan dalam pendidikan, media, teknologi	Digunakan dalam konteks agama dan budaya	Berbeza mengikut bahasa dan wilayah
Teknologi Digital	Digunakan dalam aplikasi pendidikan seperti AR Alphabets	Masih digunakan tetapi kurang dalam teknologi moden	Ada sistem khusus seperti Pinyin untuk Mandarin
Interaktiviti	Banyak aplikasi interaktif dan gamifikasi	Kurang digunakan dalam aplikasi teknologi moden	Berbeza mengikut adaptasi teknologi bahasa tersebut

\section{Aplikasi Android boleh dikembangkan pada sistem operasi berikut:}
1.Windows XP Vista atau lebih baru
2.Mac OS X atau lebih baru
3.Linux

\section{Algoritma AR Berasaskan Penanda}

Matlamat utama adalah untuk membolehkan pengguna atau pelanggan melihat objek maya dan 3D di dunia nyata menggunakan sistem AR berasaskan penanda. Pengguna boleh menyediakan imej objek, yang akan berada di semua sisi gambar objek tersebut. Imej-imej tersebut akan diletakkan dalam sebuah kubus 3D yang akan memproses dan melengkapkan objek maya tersebut.
	Pendekatan watershed berasaskan penanda adalah cara yang sangat berkesan untuk segmentasi imej dan telah banyak digunakan dalam beberapa tahun kebelakangan ini. Algoritma berasaskan penanda konvensional dilaksanakan menggunakan barisan hierarki. Satu algoritma watershed berasaskan penanda yang baru berdasarkan struktur data set terpisah dicadangkan dalam kertas ini. Ia terdiri daripada dua langkah: langkah banjir dan langkah penyelesaian. Algoritma ini mempunyai keperluan memori yang jauh lebih rendah berbanding dengan algoritma konvensional sambil mengekalkan kerumitan pengiraan O(N), di mana N adalah saiz imej. Keputusan eksperimen seterusnya menunjukkan bahawa algoritma baru yang dilaksanakan dalam bahasa C berjalan jauh lebih pantas daripada algoritma konvensional yang berdasarkan antrian hierarki hasil daripada penjimatan daripada pengalokasian dan pelepasan memori yang besar. (Gao, 2013)Pengenalan dan penentuan lokasi objek tiga dimensi (3D) adalah sangat penting dalam pelbagai aplikasi. Beberapa pelaksanaan yang berkaitan dengan pengenalan dan penentuan lokasi objek 3D yang kompleks telah dicapai menggunakan pendekatan heuristik. Kedudukan dalam ruang tiga dimensi objek tersebut masih menjadi isu yang mencabar. Oleh itu, kaedah berasaskan penanda untuk pengenalan dan penentuan lokasi objek tiga dimensi daripada imej dua dimensi mereka telah dikemukakan.

\section{SINTESIS}
Kesusasteraan dan kajian AR yang dikumpulkan daripada penulis asing dan tempatan akan digunakan untuk memahami kepentingannya, memberikan maklumat tambahan kepada penyelidik, atau berfungsi sebagai sumber untuk mempermudahkan idea dalam kajian ini. Penyelidik menyatakan bahawa AR mempunyai potensi untuk mencipta pengalaman pembelajaran yang menarik dan berkesan. Mereka menganggap AR sebagai sejenis multimedia yang terintegrasi dalam persekitaran yang autentik dan menggunakan teori pembelajaran multimedia sebagai kerangka untuk membangunkan aplikasi pendidikan ini. Mereka juga meramalkan bahawa teknologi AR akan berkembang dengan lebih pesat pada masa hadapan dan menawarkan kelebihan dalam konteks pembelajaran maya dalam pendidikan.

Para penyelidik telah melaksanakan temubual dengan seorang pekerja penjagaan kanak-kanak yang bertauliah mengenai perkembangan kanak-kanak. Dengan itu, mereka mengumpulkan maklumat tambahan melalui wawancara dengan pakar dan profesional, serta meminta pandangan individu sebagai asas untuk penambahbaikan dan cadangan bagi perkembangan kanak-kanak yang lebih praktikal. AR telah terbukti meningkatkan motivasi pembelajaran pelajar. Sehubungan itu, penyelidik tersebut telah merumuskan idea untuk membangunkan aplikasi AR yang dinamakan AR Alphabets.


\section{UML (Unified Modeling Language )}
Dengan perkembangan teknik pengaturcaraan berorientasikan objek, bahasa pemodelan piawai telah muncul untuk membina perisian yang dibangunkan dengan menggunakan teknik pengaturcaraan berorientasikan objek. Unified Modeling Language (UML) muncul sebagai respons kepada keperluan untuk pemodelan visual yang mendefinisikan, menerangkan, membina, dan mendokumentasikan sistem perisian. Unified Modeling Language (UML) berfungsi sebagai bahasa visual untuk memodelkan dan berkomunikasi mengenai sistem melalui penggunaan rajah dan teks sokongan. Untuk memperjelas reka bentuk aplikasi yang dibangunkan, UML telah diterapkan, yang merangkumi pelbagai jenis rajah:Adalah salah satu kaedah pemodelan visual  digunakan dalam reka bentuk dan  membuat perisian yang  berfokus pada objek. UML adalah sebuah bahasa untuk memodelkan perisian yang  disandarkan sebagai medium penulisan pelan perisian  (terbitan) (Sumiati et al., 2021). UML  adalah piawaian penulisan atau  sejenis pelan yang termasuk ermasuk proses perniagaan, menulis kelas - kelas dalam bahasa tertentu. Terdapat beberapa rajah UML yang sering digunakan dalam pembangunan sebuah sistem, iaitu: 

Jadual  2-9  Komponen dan Penggunaan UML
Jenis Diagram UML	Fungsi Utama	Contoh Penggunaan
Use Case Diagram	Mewakili interaksi antara pengguna dan sistem	Merancang fungsi utama dalam aplikasi pendidikan
Class Diagram	Menunjukkan struktur dan hubungan antara kelas dalam sistem	Reka bentuk objek dan struktur data dalam aplikasi AR
Sequence Diagram	Memaparkan aliran interaksi antara objek dalam urutan masa	Menganalisis proses interaksi pengguna dengan sistem
Activity Diagram	Mewakili aliran kerja atau logik proses	Menggambarkan langkah pembelajaran interaktif dalam aplikasi
State Diagram	Menunjukkan keadaan objek dalam sistem dan perubahannya	Memodelkan status perubahan antara modul dalam aplikasi AR Alphabets
Component Diagram	Mewakili komponen perisian dan hubungan antara mereka	Memetakan struktur arkitektur sistem AR
Deployment Diagram	Menggambarkan bagaimana komponen perisian disusun dalam persekitaran fizikal	Konfigurasi sistem AR dalam platform pendidikan


