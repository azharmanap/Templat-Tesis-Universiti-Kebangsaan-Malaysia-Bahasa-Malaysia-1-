\chapter{Pengenalan}
\label{bab:pengenalan}

\section{Pendahuluan}
Teknologi Augmented Reality (AR) telah mengalami evolusi yang signifikan, terutamanya dalam sektor pendidikan, kerana kemampuannya menggabungkan elemen maya dan nyata dalam satu ruang interaktif yang mampu meningkatkan keberkesanan pembelajaran (Azuma, 1997; Wu et al., 2013; Bacca et al., 2014; Billinghurst \& Dünser, 2012). Dalam pendidikan prasekolah, asas literasi dilihat sebagai tunjang utama dalam membentuk kemahiran asas membaca dan menulis yang kritikal untuk perkembangan kognitif kanak-kanak (Mayo, 2019; Jamila et al., 2012). Beberapa kajian menunjukkan bahawa penggunaan pendekatan pembelajaran berasaskan visual, auditori, dan interaktif — seperti yang disokong oleh teknologi AR — dapat memudahkan pemahaman konsep asas huruf serta meningkatkan motivasi dan penglibatan murid (Chen et al., 2020; Rahmawati et al., 2022; Bower et al., 2014; Radu, 2014).
Pendekatan tradisional dalam pendidikan awal, seperti penggunaan buku teks, kad imbas, dan latihan bertulis, masih menjadi pilihan utama di kebanyakan prasekolah. Namun, kurangnya unsur interaktif dalam kaedah konvensional ini sering menyebabkan sebilangan murid menghadapi kesukaran dalam mengenal pasti dan menguasai huruf dengan berkesan (Rahmawati et al., 2022; Jamila et al., 2012). Sebaliknya, aplikasi teknologi AR memberikan peluang pembelajaran yang lebih dinamik dan menyeronokkan dengan membolehkan murid memvisualisasikan huruf dalam bentuk tiga dimensi (3D), mendengar sebutan yang betul, serta berinteraksi dengan animasi yang dapat memperkukuhkan pemahaman konsep (Gunalan et al., 2023; Billinghurst \& Dünser, 2012). Oleh yang demikian, integrasi AR dalam proses pengajaran dan pembelajaran bukan sahaja dapat merangsang minat dan motivasi murid, tetapi juga meningkatkan daya ingatan serta penguasaan literasi awal melalui pendedahan visual, auditori, dan elemen interaktif (UNESCO, 2022; Chen et al., 2020; Radu, 2014).
Kajian ini memfokuskan kepada penilaian keberkesanan aplikasi AR Alphabets dalam membantu murid prasekolah mengenal huruf dengan cara yang lebih menyeronokkan dan interaktif, di samping meneliti kelebihan serta kekurangan penggunaan teknologi ini berbanding pendekatan tradisional seperti buku teks dan latihan bertulis (Kementerian Pendidikan Malaysia, 2015; Gunalan et al., 2023). Melalui integrasi elemen visual, audio, dan animasi dalam aplikasi AR, diharapkan pembelajaran huruf menjadi lebih efektif serta dapat meningkatkan pemahaman dan minat murid (Billinghurst & Dünser, 2012; Wu et al., 2013). Dapatan kajian lepas turut membuktikan bahawa penglibatan aktif murid dalam aktiviti pembelajaran yang menyeronokkan dan interaktif mampu mempercepat penguasaan literasi awal (Mayo, 2019; Bower et al., 2014). Sejajar dengan hasrat untuk memperkasakan pendidikan abad ke-21, Kementerian Pendidikan Malaysia (KPM) turut menyokong penggunaan teknologi inovatif seperti AR di peringkat prasekolah melalui Pelan Pembangunan Pendidikan Malaysia (PPPM) 2013–2025 (Kementerian Pendidikan Malaysia, 2013).
Menurut Kementerian Pendidikan Malaysia (KPM), penggunaan teknologi AR dalam pendidikan prasekolah dapat membantu murid memahami bentuk serta bunyi huruf dengan lebih jelas melalui gabungan visualisasi dan audio yang interaktif. Pendekatan ini juga mampu merangsang perkembangan kemahiran kognitif dan motor halus melalui aktiviti sentuhan serta manipulasi objek digital, selain menggalakkan pembelajaran kendiri di mana kanak-kanak dapat meneroka huruf dalam suasana pembelajaran yang menyeronokkan dan motivasi tinggi (Kementerian Pendidikan Malaysia, 2015; Gunalan et al., 2023). Tambahan pula, laporan UNESCO MGIEP menegaskan bahawa teknologi AR dapat meningkatkan tumpuan dan daya ingatan murid kerana persembahan konsep secara visual dan interaktif mendorong pelajar untuk lebih mudah mengingati serta memahami maklumat (UNESCO, 2022; Bower et al., 2014; Radu, 2014).
Kajian ini bertujuan membangunkan serta menilai aplikasi AR Alphabets dalam konteks pembelajaran prasekolah, di samping meneliti bagaimana teknologi inovatif ini dapat menyokong murid dalam mengenali dan memahami huruf secara lebih berkesan. Kementerian Pendidikan Malaysia (KPM) telah mengambil langkah proaktif dalam mereformasi sistem pendidikan negara, termasuk menerusi pelaksanaan inisiatif seperti Mesyuarat Susulan Jemaah Menteri Bil. 6/2008 dan Pelan Pembangunan Pendidikan Malaysia (PPPM) 2013–2025, yang menekankan kepentingan penggunaan teknologi digital dalam pendidikan (Kementerian Pendidikan Malaysia, 2013). Sejajar dengan tuntutan Revolusi Industri Keempat (IR 4.0), KPM memperkenalkan Pendidikan 4.0 yang memberi fokus kepada penguasaan kemahiran abad ke-21, antaranya pemikiran kritis, kreativiti, penyelesaian masalah dan penerapan pembelajaran berasaskan teknologi (Kementerian Pendidikan Malaysia, 2015; UNESCO, 2022; Gunalan et al., 2023)..
Pengaplikasian teknologi seperti Augmented Reality (AR) semakin diiktiraf sebagai pemangkin pemodenan dalam pendidikan, di mana guru dapat menyampaikan kandungan pengajaran dengan lebih menarik, interaktif, dan efektif. Inisiatif ini selari dengan dasar ICT dalam pendidikan negara yang menekankan penggunaan teknologi digital sebagai strategi meningkatkan kualiti dan akses kepada pembelajaran (Kementerian Pendidikan Malaysia, 2015; Gunalan et al., 2023; Bower et al., 2014). Meskipun terdapat cabaran seperti kos pembangunan aplikasi AR serta keperluan latihan khusus untuk guru, teknologi ini tetap berpotensi besar dalam memperkasakan sistem pendidikan Malaysia agar lebih moden, responsif dan inklusif, selaras dengan aspirasi Pendidikan 4.0 (Gunalan et al., 2023; UNESCO, 2022).
1.2Latar Belakang Kajian
Pembelajaran literasi awal merupakan asas utama dalam perkembangan akademik dan kognitif kanak-kanak, terutamanya bagi murid prasekolah yang sedang belajar mengenali huruf dan perkataan (Mayo, 2019; Zhou et al., 2021). Kajian menunjukkan bahawa kemahiran membaca dan menulis yang kukuh di peringkat prasekolah berkait rapat dengan prestasi akademik di sekolah rendah dan seterusnya (Zhou et al., 2021).
Pendekatan tradisional dalam pembelajaran huruf, seperti penggunaan buku teks, kad imbas, dan kaedah pengulangan, masih digunakan dalam sistem pendidikan. Walau bagaimanapun, kaedah ini mungkin kurang menarik bagi murid prasekolah dan boleh menyebabkan mereka hilang fokus semasa belajar (Chen et al., 2020). Oleh itu, integrasi teknologi seperti Augmented Reality (AR) menawarkan pendekatan pembelajaran yang lebih interaktif dan sesuai dengan perkembangan digital dalam pendidikan awal kanak-kanak.
Teknologi AR membolehkan pengguna berinteraksi dengan objek digital dalam dunia nyata, menjadikan pengalaman pembelajaran lebih visual, dinamik, dan menarik (Azuma, 1997). Kajian telah menunjukkan bahawa penggunaan AR dalam pendidikan mampu meningkatkan pemahaman murid, motivasi, dan daya ingatan, serta membolehkan mereka mengalami konsep pembelajaran dengan lebih realistik (Wu et al., 2013).
Kajian oleh Billinghurst dan Dünser (2012) mendapati bahawa murid yang belajar menggunakan modul interaktif berasaskan AR mampu mengingati konsep dengan lebih cepat berbanding mereka yang menggunakan bahan pembelajaran tradisional. Sementara itu, Rahmawati et al. (2022) menunjukkan bahawa AR dapat membantu kanak-kanak mengenali huruf dengan lebih efektif melalui penggunaan model 3D dan kesan animasi.Di Malaysia, KPM menggalakkan penggunaan AR dalam pendidikan prasekolah, selaras dengan usaha memperkukuh literasi digital generasi muda (KPM, 2013).
Penggunaan AR dalam Pendidikan Prasekolah di Malaysia
Di Malaysia, Kementerian Pendidikan Malaysia (KPM) telah menggalakkan penggunaan teknologi digital dalam pendidikan prasekolah, termasuk elemen gamifikasi, AR, dan multimedia interaktif, sejajar dengan Pelan Pembangunan Pendidikan Malaysia (PPPM) 2013–2025.
Menurut laporan KPM, penggunaan AR dalam pendidikan prasekolah boleh membantu murid untuk memahami bentuk dan bunyi huruf dengan lebih jelas melalui visualisasi dan audio interaktif, meningkatkan kemahiran kognitif dan motor halus dengan aktiviti sentuhan serta manipulasi objek huruf dalam AR, serta menggalakkan pembelajaran kendiri yang membolehkan kanak-kanak meneroka huruf secara lebih menyeronokkan.
Selain itu, laporan UNESCO MGIEP menyatakan bahawa teknologi AR dapat meningkatkan tumpuan dan daya ingatan pelajar, kerana mereka lebih cenderung mengingati sesuatu konsep apabila ia dipersembahkan dalam bentuk visual dan interaktif (UNESCO, 2022).
Fokus Kajian
Berdasarkan maklumat di atas, kajian ini akan membangunkan dan menilai keberkesanan aplikasi AR Alphabets dalam membantu murid prasekolah mengenali huruf dengan pendekatan yang lebih interaktif dan menyeronokkan.
Kajian ini juga akan menggunakan pengujian kebolehgunaan seperti System Usability Scale (SUS) untuk menilai kemudahan penggunaan aplikasi serta pengalaman pengguna.
1.3Pernyataan Masalah
Pembelajaran literasi awal merupakan asas penting dalam pendidikan prasekolah kerana ia membantu kanak-kanak mengenali huruf, memahami bunyi, dan mengembangkan kemahiran membaca. Walaupun terdapat pelbagai inovasi teknologi pendidikan, kaedah konvensional masih menjadi pilihan utama di peringkat prasekolah. Kajian menunjukkan murid prasekolah menghadapi cabaran dalam pembelajaran literasi disebabkan kekurangan elemen interaktif dan motivasi yang rendah (Rahmawati et al., 2022; Zhou et al., 2021).
Guru juga menghadapi keterbatasan dari segi latihan penggunaan teknologi seperti AR (UNESCO, 2022). Justeru, pembangunan aplikasi AR Alphabets ini diharap dapat menambah nilai dan meningkatkan keberkesanan pembelajaran literasi awa.
Beberapa isu utama yang dikenal pasti dalam pembelajaran huruf bagi kanak-kanak prasekolah adalah seperti berikut:
1.Kurangnya elemen interaktif dalam pembelajaran huruf.
2.Motivasi pembelajaran yang rendah dalam kalangan murid prasekolah.
3.Kesukaran mengingat bentuk dan bunyi huruf, terutama bagi huruf yang mempunyai bentuk hampir serupa (contoh: “b” dan “d”).
4.Keterbatasan guru dalam menerapkan teknologi pendidikan, kerana tidak semua guru diberi latihan yang mencukupi untuk menggunakan alat pembelajaran digital seperti AR (UNESCO, 2022).
Kajian menunjukkan bahawa kaedah pembelajaran berasaskan visual dan auditori dapat membantu meningkatkan kefahaman murid (Billinghurst & Dünser, 2012). Murid prasekolah sering menghadapi cabaran dalam mengenal pasti bentuk huruf serta mengingati bunyi huruf dengan betul. Selain itu, kajian mendapati bahawa pelajar lebih cenderung untuk hilang fokus dalam pembelajaran huruf apabila tiada elemen interaktif dan menarik, menyebabkan mereka lambat dalam proses pengecaman huruf dan sebutan (Rahmawati et al., 2022).
Sebagai penyelesaian kepada masalah ini, teknologi Augmented Reality (AR) menawarkan pendekatan pembelajaran yang lebih visual, interaktif, dan menarik. AR dapat membantu murid prasekolah melihat, mendengar, dan berinteraksi dengan huruf dalam bentuk 3D, menjadikan pengalaman pembelajaran lebih menyeronokkan dan mudah difahami.
Kajian ini bertujuan untuk menilai keberkesanan aplikasi AR Alphabets dalam meningkatkan pengalaman pembelajaran huruf bagi murid prasekolah, serta mengenal pasti keuntungan dan cabaran teknologi ini berbanding kaedah pembelajaran tradisional. Murid prasekolah cenderung belajar dengan menggunakan deria mereka, namun kaedah pembelajaran tradisional kurang menawarkan visualisasi dinamik, animasi, dan elemen auditori yang dapat membantu mereka mengenali huruf dengan lebih berkesan (Chen et al., 2020).
1.4Objektif Kajian
Kajian ini bertujuan untuk:
1.Mengkaji pencapaian murid setelah menjalani proses pembelajaran menggunakan aplikasi AR Alphabets, khususnya dalam pengenalan huruf dan pemahaman fonetik. 
2.Menilai perubahan pemahaman murid sebelum dan selepas penggunaan AR Alphabets, bagi mengenal pasti sejauh mana aplikasi ini membantu murid memahami huruf secara lebih berkesan melalui elemen interaktif seperti audio dan animasi 3D.
3.Menganalisis perubahan daya tumpuan murid selepas proses pembelajaran menggunakan AR Alphabets, dan penerimaan pengguna terhadap aplikasi AR Alphabets

1.5Soalan Kajian
Persoalan kajian adalah seperti berikut :
1.Adakah terdapat peningkatan prestasi pencapaian murid setelah menjalani proses pembelajaran menggunakan aplikasi AR Alphabets?
2.Adakah terdapat perubahan dalam pemahaman murid setelah melalui proses pembelajaran menggunakan AR Alphabets, khususnya dalam mengenal huruf dan memahami fonetik?
3.Adakah terdapat perubahan dalam tumpuan murid selepas menggunakan AR Alphabets, dan apakah tahap penerimaan guru dan murid terhadap AR Alphabets ?

1.6Batasan Kajian
Kajian ini menumpukan kepada penggunaan aplikasi AR Alphabets dalam pembelajaran literasi awal bagi murid prasekolah di sebuah institusi pendidikan prasekolah yang dipilih.
Struktur organisasi prasekolah yang menjadi tempat kajian terdiri daripada seorang guru besar, seorang penolong kanan pentadbiran, seorang penolong kanan hal ehwal murid, dan seorang penolong kanan kokurikulum. Institusi tersebut mempunyai jumlah keseluruhan murid prasekolah, dengan sekumpulan murid yang dipilih sebagai sampel kajian berdasarkan pengalaman mereka dalam pembelajaran literasi awal.
Kajian ini tidak melibatkan murid pendidikan khas, tetapi memberi tumpuan kepada murid prasekolah yang mengikuti kurikulum biasa, khususnya dalam pembelajaran mengenal huruf dan memahami bunyi fonetik menggunakan teknologi Augmented Reality (AR).
Saiz sampel kajian terdiri daripada sekumpulan murid prasekolah yang dipilih berdasarkan interaksi mereka dengan AR Alphabets, bagi menilai impak aplikasi terhadap pemahaman huruf, daya tumpuan, dan motivasi pembelajaran mereka.
Kajian ini terhad kepada satu institusi prasekolah, dan penemuan yang diperoleh akan memberi gambaran tentang keberkesanan AR dalam pendidikan awal, namun tidak boleh digeneralisasikan kepada semua sekolah prasekolah di Malaysia tanpa kajian lanjut.
1.7 Kepentingan Kajian
Kajian ini mempunyai kepentingan yang besar dalam bidang pendidikan prasekolah, khususnya dalam pembelajaran literasi awal menggunakan teknologi Augmented Reality (AR).
1.7.1 Kepentingan kepada Murid Prasekolah
Meningkatkan pemahaman dan daya ingatan murid terhadap bentuk dan bunyi huruf melalui visualisasi interaktif.  Menggalakkan pembelajaran kendiri, membolehkan murid berinteraksi dengan huruf dalam bentuk 3D dan memahami konsep secara aktif. Menjadikan pembelajaran lebih menyeronokkan dan menarik, membantu meningkatkan motivasi murid dalam mengenali huruf dengan lebih cepat.(Wu et al., 2013)
1.7.2 Kepentingan kepada Guru
Membantu guru dalam menyampaikan pelajaran dengan lebih efektif, menggunakan animasi 3D dan bunyi sebutan huruf. Menyediakan alat bantu mengajar yang inovatif, yang boleh digunakan untuk meningkatkan keberkesanan pengajaran literasi awal. (Gunalan et al., 2023)Memudahkan guru mengenal pasti kesulitan murid dalam pembelajaran huruf, dengan adanya sistem interaktif dan maklum balas digital.
1.7.3Kepentingan kepada Sistem Pendidikan
Menyokong Pelan Pembangunan Pendidikan Malaysia (PPPM) 2013-2025, yang menggalakkan penggunaan teknologi dalam pendidikan. Membantu memperkaya kurikulum pendidikan prasekolah, dengan mengintegrasikan teknologi digital dan pembelajaran interaktif. (KPM, 2013; UNESCO, 2022)Menjadi rujukan kepada kajian teknologi pendidikan, khususnya dalam pengembangan aplikasi pembelajaran berbasis AR bagi kanak-kanak.
1.7.4 Kepentingan kepada Penyelidikan Teknologi
Menyumbang kepada inovasi teknologi pendidikan, dengan membangunkan aplikasi AR Alphabets yang lebih mesra pengguna (Bacca et al., 2014). Membantu dalam memahami keberkesanan AR dalam literasi awal, dengan menggunakan metodologi pengujian usability seperti System Usability Scale (SUS). Menjadi asas kepada kajian lanjut dalam bidang AR, khususnya dalam pembangunan aplikasi pendidikan interaktif untuk murid prasekolah.
1.8Definisi Operasi
1.8.1Augmented Reality (AR)
Definisi Umum: Teknologi yang menggabungkan elemen digital ke dalam dunia nyata, membolehkan pengguna berinteraksi dengan objek maya dalam persekitaran fizikal  (Azuma, 1997; Wu et al., 2013). Definisi Operasi dalam Kajian Ini: AR digunakan dalam aplikasi AR Alphabets untuk membantu murid prasekolah melihat, mendengar, dan berinteraksi dengan huruf dalam bentuk 3D bagi meningkatkan pemahaman mereka terhadap literasi awal.
1.8.2Literasi Awal
Definisi Umum: Keupayaan kanak-kanak untuk mengenali huruf, memahami bunyi, dan mengembangkan kemahiran membaca serta menulis (Mayo, 2019). Definisi Operasi dalam Kajian Ini: Literasi awal merujuk kepada kemampuan murid prasekolah mengenali dan mengingat bentuk serta bunyi huruf, yang diuji melalui penggunaan aplikasi AR Alphabets.
1.8.3Murid Prasekolah
Definisi Umum: Kanak-kanak berusia 4 hingga 6 tahun yang berada dalam fasa pendidikan awal sebelum memasuki sekolah rendah (UNESCO, 2022). �� Definisi Operasi dalam Kajian Ini: Murid prasekolah yang terlibat dalam kajian ini berusia 5 hingga 6 tahun, di mana mereka diuji untuk melihat keberkesanan penggunaan AR dalam pembelajaran huruf.
1.8.4Aplikasi Pembelajaran
Aplikasi pembelajaran ialah perisian yang dibangunkan untuk menyokong proses pengajaran dan pembelajaran (PdP). Dalam kajian ini, aplikasi AR Alphabets dibangunkan sebagai bahan bantu mengajar (BBM) dalam pembelajaran literasi awal, membolehkan murid mengimbas kad huruf dan melihat animasi interaktif sebagai sebahagian daripada kaedah pembelajaran digital yang lebih menarik.
1.8.5Pencapaian
Pencapaian akademik dalam kajian ini merujuk kepada kemampuan murid mengenal huruf dan memahami fonetik selepas menggunakan AR Alphabets. Kajian menilai kemajuan murid melalui ujian pra dan ujian pasca, bagi melihat sejauh mana aplikasi ini membantu mereka mengenal pasti huruf dengan lebih berkesan.
1.8.6 Pemahaman
Pemahaman dalam konteks kajian ini ialah keupayaan murid untuk mengenali huruf dan memahami bunyi fonetik dengan lebih mendalam, berdasarkan visualisasi yang diberikan dalam aplikasi AR Alphabets.
1.8.7Tumpuan
Tumpuan merujuk kepada kemampuan murid untuk memberi perhatian dalam proses pembelajaran. Kajian ini mengkaji sejauh mana AR Alphabets membantu meningkatkan daya fokus murid terhadap literasi awal, berbanding kaedah pengajaran tradisional yang kurang interaktif.



?}
\label{sec:apadia}

Lorem Ipsum adalah contoh teks atau dummy dalam industri percetakan dan penataan huruf atau typesetting. Lorem Ipsum telah menjadi standar contoh teks sejak tahun 1500an, saat seorang tukang cetak yang tidak dikenal mengambil sebuah kumpulan teks dan mengacaknya untuk menjadi sebuah buku contoh huruf \cite{banerjee:pedersen:2003}. Ia tidak hanya bertahan selama 5 abad, tapi juga telah beralih ke penataan huruf elektronik, tanpa ada perubahan apapun. Ia mulai dipopulerkan pada tahun 1960 dengan diluncurkannya lembaran-lembaran Letraset yang menggunakan kalimat-kalimat dari Lorem Ipsum, dan seiring munculnya perangkat lunak Desktop Publishing seperti Aldus PageMaker juga memiliki versi Lorem Ipsum \cite{berment:phd:2004}.



\section{Dari Mana Asalnya?}
\label{sec:asal}

\begin{equation}
C_p = \frac{P - P_\infty}{\frac{1}{2} \rho {U_\infty}^2}
 = 1 - \left( \frac{U_1}{U_\infty} \right)^2,
\end{equation}

Tidak seperti anggapan banyak orang, Lorem Ipsum bukanlah teks-teks yang diacak. Ia berakar dari sebuah naskah sastra latin klasik dari era 45 sebelum masehi, hingga bisa dipastikan usianya telah mencapai lebih dari 2000 tahun. Richard McClintock, seorang professor Bahasa Latin dari Hampden-Sidney College di Virginia, mencoba mencari makna salah satu kata latin yang dianggap paling tidak jelas, yakni consectetur, yang diambil dari salah satu bagian Lorem Ipsum. Setelah ia mencari maknanya di di literatur klasik, ia mendapatkan sebuah sumber yang tidak bisa diragukan. Lorem Ipsum berasal dari bagian 1.10.32 dan 1.10.33 dari naskah ``de Finibus Bonorum et Malorum'' (Sisi Ekstrim dari Kebaikan dan Kejahatan) karya Cicero, yang ditulis pada tahun 45 sebelum masehi \cite{azarova:etal:2002,budanitsky:hirst:2006}. Buku ini adalah risalah dari teori etika yang sangat terkenal pada masa Renaissance. Baris pertama dari Lorem Ipsum, ``Lorem ipsum dolor sit amet\ldots'', berasal dari sebuah baris di bagian 1.10.32.

Bagian standar dari teks Lorem Ipsum yang digunakan sejak tahun 1500an kini di reproduksi kembali di bawah ini untuk mereka yang tertarik. Bagian 1.10.32 dan 1.10.33 dari "de Finibus Bonorum et Malorum" karya Cicero juga di reproduksi persis seperti bentuk aslinya, diikuti oleh versi bahasa Inggris yang berasal dari terjemahan tahun 1914 oleh H. Rackham.

\begin{equation}
-\frac{(x_0 - \mu)^2}{2 \sigma^2} = -\ln 2
\end{equation}


\section{Contoh}
\label{sec:contol}

Perenggan awal Lorem Ipsum seperti di bawah.

\subsection{Perenggan Pertama}

Lorem ipsum dolor sit amet, consectetur adipiscing elit. Donec posuere, neque quis feugiat egestas, quam sapien dictum justo, eu vulputate nunc metus sed dui. Integer molestie leo quis libero facilisis, dictum pretium quam ornare. Vestibulum ante ipsum primis in faucibus orci luctus et ultrices posuere cubilia Curae; Vivamus luctus rutrum magna non convallis. Praesent vestibulum consequat eros, et fringilla nisi suscipit id. Nam vulputate justo dui, eu rutrum est accumsan ut. Sed molestie erat vitae mi blandit, in volutpat urna lobortis. Vestibulum mollis rutrum gravida. Fusce dolor nulla, condimentum vel pretium ut, venenatis eget leo. Ut semper placerat mauris, ut tempus est tempor vel. Interdum et malesuada fames ac ante ipsum primis in faucibus. In vitae feugiat diam. Pellentesque accumsan consequat turpis aliquam elementum.


\subsection{Dua Perenggan Seterusnya}

Vivamus dignissim arcu nunc, non aliquam sem porta vitae. Sed sodales accumsan dui sit amet egestas. Maecenas rhoncus a erat eget accumsan.

\begin{table}[hbt!]\centering
\caption{Bilangan permata}

\begin{tabular}{l c}
\hline
Jenis & Bilangan \\\hline
Nilam & 6\\
Berlian & 23\\
Emas & 56\\
Perak & 235\\
Gangsa & 324\\\hline
\end{tabular}
\end{table}

\begin{itemize}
\item Etiam vitae pulvinar metus, sed fringilla orci.
\item Duis dapibus dolor risus, non ultrices enim porta sit amet.
\item Ut eu libero augue.
\end{itemize}

Nulla ipsum augue, feugiat ac laoreet quis, pretium ut magna. Class aptent taciti sociosqu ad litora torquent per conubia nostra, per inceptos himenaeos. Integer blandit placerat dictum.

\begin{figure}[hbt!]\centering
\includegraphics[width=.5\textwidth]{green}
\caption{Contoh rajah}
\end{figure}

Sed dolor justo, scelerisque sed rutrum quis, porttitor a mauris. Cras non auctor felis, rutrum fringilla risus. Integer at convallis erat, sit amet luctus turpis. Duis sed rutrum eros, quis tempus risus. Etiam pellentesque nisi odio, eget dignissim eros ultrices et. Aliquam leo massa, fermentum vel odio sed, ullamcorper molestie lorem. Integer lorem felis, adipiscing sit amet interdum eget, auctor at lorem. Aliquam ultricies tortor eu nibh facilisis tincidunt.


\subsubsection{Sedikit Catatan}

Duis sed rutrum eros, quis tempus risus. Etiam pellentesque nisi odio, eget dignissim eros ultrices et. Aliquam leo massa, fermentum vel odio sed, ullamcorper molestie lorem.

\subsubsection{Selanjutnya}
Duis sed rutrum eros, quis tempus risus. Etiam pellentesque nisi odio, eget dignissim eros ultrices et. Aliquam leo massa, fermentum vel odio sed, ullamcorper molestie lorem.


\section{Ringkasan}
Nulla ipsum augue, feugiat ac laoreet quis, pretium ut magna. Class aptent taciti sociosqu ad litora torquent per conubia nostra, per inceptos himenaeos. Integer blandit placerat dictum.

Sed dolor justo, scelerisque sed rutrum quis, porttitor a mauris. Cras non auctor felis, rutrum fringilla risus. Integer at convallis erat, sit amet luctus turpis. Duis sed rutrum eros, quis tempus risus. Etiam pellentesque nisi odio, eget dignissim eros ultrices et. Aliquam leo massa, fermentum vel odio sed, ullamcorper molestie lorem. Integer lorem felis, adipiscing sit amet interdum eget, auctor at lorem. Aliquam ultricies tortor eu nibh facilisis tincidunt.
