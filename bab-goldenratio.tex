\chapter{Golden Ratio}

In mathemat
\chapter{Perbincangan, Implikasi dan Cadangan}

\section{Pengenalan}
Bab ini membincangkan dapatan kajian secara menyeluru
\documentclass[12pt]{report}
\usepackage[utf8]{inputenc}
\usepackage{geometry}
\geometry{a4paper, margin=1in}
\usepackage{setspace}
\usepackage{lipsum}
\usepackage{titlesec}
\usepackage{hyperref}
\usepackage{graphicx}
\usepackage{enumitem}

\titleformat{\chapter}[display]
  {\normalfont\huge\bfseries}{\chaptertitlename\ \thechapter}{20pt}{\Huge}

\title{Pembangunan Aplikasi AR Alphabets Prasekolah}
\author{Azhar Manap}
\date{}

\begin{document}

\maketitle
\tableofcontents
\newpage

\chapter{Perbincangan, Implikasi dan Cadangan}

\section{Pengenalan}
Bab ini membincangkan dapatan kajian secara menyeluruh berdasarkan objektif kajian yang telah ditetapkan. Perbincangan ini turut merangkumi implikasi kajian terhadap bidang pendidikan prasekolah, khususnya dalam aspek pengajaran literasi awal menggunakan teknologi Augmented Reality (AR). Selain itu, cadangan bagi kajian lanjutan turut dikemukakan.

\section{Perbincangan Dapatan Kajian}
\subsection{Keberkesanan Aplikasi AR Alphabets terhadap Pencapaian Literasi Awal}
Dapatan menunjukkan bahawa penggunaan aplikasi AR Alphabets telah memberikan impak positif terhadap penguasaan huruf dan fonetik dalam kalangan murid prasekolah...

\subsection{Pengaruh Aplikasi terhadap Tumpuan dan Minat Murid}
Aplikasi AR Alphabets berjaya mengekalkan tumpuan murid dalam jangka masa lebih panjang...

\subsection{Kebolehgunaan Aplikasi oleh Guru dan Murid}
Ujian kebolehgunaan menunjukkan skor tinggi pada Skala Kebolehgunaan Sistem (SUS)...

\section{Implikasi Kajian}
\subsection{Implikasi terhadap Amalan Pengajaran}
Kajian ini menunjukkan bahawa penggunaan teknologi AR bukan sahaja berupaya meningkatkan keberkesanan pengajaran dan pembelajaran...

\subsection{Implikasi terhadap Dasar Pendidikan}
Hasil kajian menyokong pelaksanaan inisiatif transformasi digital dalam pendidikan seperti dalam PPPM 2013–2025...

\subsection{Implikasi terhadap Penyelidikan Masa Hadapan}
Kajian ini membuka ruang kepada lebih banyak kajian lanjutan dalam bidang pembangunan teknologi pendidikan awal kanak-kanak...

\section{Kekuatan dan Kelemahan Kajian}
Kajian ini berjaya membangunkan aplikasi yang berkesan dan diterima baik oleh pengguna...

\section{Cadangan Kajian Lanjutan}
\begin{itemize}
  \item Memperluaskan penggunaan aplikasi ke lebih banyak institusi prasekolah
  \item Membangunkan modul tambahan seperti ejaan dan suku kata
  \item Menyesuaikan aplikasi ke pelantar lain seperti iOS
  \item Mengintegrasikan AI untuk pembelajaran adaptif
\end{itemize}

\section{Penutup}
Bab ini membincangkan dapatan utama kajian dan cadangan penambahbaikan bagi meningkatkan keberkesanan aplikasi dan memperluaskan impaknya.

\end{document}
h berdasarkan objektif kajian yang telah ditetapkan. Perbincangan ini turut merangkumi implikasi kajian terhadap bidang pendidikan prasekolah, khususnya dalam aspek pengajaran literasi awal menggunakan teknologi Augmented Reality (AR). Selain itu, cadangan bagi kajian lanjutan turut dikemukakan.

\section{Perbincangan Dapatan Kajian}
\subsection{Keberkesanan Aplikasi AR Alphabets terhadap Pencapaian Literasi Awal}
Dapatan menunjukkan bahawa penggunaan aplikasi AR Alphabets telah memberikan impak positif terhadap penguasaan huruf dan fonetik dalam kalangan murid prasekolah. Murid dapat mengenal huruf dengan lebih cepat serta memahami bunyi huruf melalui pendekatan interaktif dan visual yang ditawarkan oleh aplikasi tersebut...

\subsection{Pengaruh Aplikasi terhadap Tumpuan dan Minat Murid}
Aplikasi AR Alphabets berjaya mengekalkan tumpuan murid dalam jangka masa lebih panjang...

\subsection{Kebolehgunaan Aplikasi oleh Guru dan Murid}
Ujian kebolehgunaan menunjukkan skor tinggi pada Skala Kebolehgunaan Sistem (SUS)...

\section{Implikasi Kajian}
\subsection{Implikasi terhadap Amalan Pengajaran}
Kajian ini menunjukkan bahawa penggunaan teknologi AR bukan sahaja berupaya meningkatkan keberkesanan pengajaran dan pembelajaran...

\subsection{Implikasi terhadap Dasar Pendidikan}
Hasil kajian menyokong pelaksanaan inisiatif transformasi digital dalam pendidikan seperti dalam PPPM 2013–2025...

\subsection{Implikasi terhadap Penyelidikan Masa Hadapan}
Kajian ini membuka ruang kepada lebih banyak kajian lanjutan dalam bidang pembangunan teknologi pendidikan awal kanak-kanak...

\section{Kekuatan dan Kelemahan Kajian}
Kajian ini berjaya membangunkan aplikasi yang berkesan dan diterima baik oleh pengguna...

\section{Cadangan Kajian Lanjutan}
\begin{itemize}
  \item Memperluaskan penggunaan aplikasi ke lebih banyak institusi prasekolah
  \item Membangunkan modul tambahan seperti ejaan dan suku kata
  \item Menyesuaikan aplikasi ke pelantar lain seperti iOS
  \item Mengintegrasikan AI untuk pembelajaran adaptif
\end{itemize}

\section{Penutup}
Bab ini membincangkan dapatan utama kajian dan cadangan penambahbaikan bagi meningkatkan keberkesanan aplikasi dan memperluaskan impaknya.
ics and the arts, two quantities are in the golden ratio if their ratio is the same as the ratio of their sum to the larger of the two quantities, i.e.~their maximum. The figure on the right illustrates the geometric relationship. Expressed algebraically, for quantities $a$ and $b$ with $a > b$,
\begin{equation}
 \frac{a+b}{a} = \frac{a}{b} \ \stackrel{\text{def}}{=}\ \varphi,
\end{equation}
where the Greek letter $\varphi$ represents the golden ratio. Its value is:
\begin{equation}
\varphi = \frac{1+\sqrt{5}}{2} = 1.61803\,39887\ldots.
\end{equation}

\section{History}

Ancient Greek mathematicians first studied what we now call the golden ratio because of its frequent appearance in geometry. The division of a line into ``extreme and mean ratio'' (the golden section) is important in the geometry of regular pentagrams and pentagons. Euclid's Elements  provides the first known written definition of what is now called the golden ratio: ``A straight line is said to have been cut in extreme and mean ratio when, as the whole line is to the greater segment, so is the greater to the less.'' Euclid explains a construction for cutting (sectioning) a line ``in extreme and mean ratio'', i.e., the golden ratio. (See Figure~\ref{fig:line:golden}.) Throughout the Elements, several propositions (theorems in modern terminology) and their proofs employ the golden ratio.

\begin{figure}[hbt!]\centering
\includegraphics[width=.3\textwidth]{220px-Golden-ratio-line}
\caption{Line segments in the golden ratio}
\label{fig:line:golden}
\end{figure}

\begin{figure}[hbt!]\centering
\includegraphics[width=.3\textwidth]{SimilarGoldenRectangles}
\caption{Golden rectangles}
\end{figure}



\section{Calculation}
Two quantities $a$ and $b$ are said to be in the golden ratio $\varphi$ if:
\begin{equation}
 \frac{a+b}{a} = \frac{a}{b} = \varphi.
\end{equation}

One method for finding the value of $\varphi$ is to start with the left fraction. Through simplifying the fraction and substituting in $\frac{b}{a} = \frac{1}{\varphi}$,
\begin{equation}
\frac{a+b}{a} = 1 + \frac{b}{a} = 1 + \frac{1}{\varphi},
\end{equation}

By definition, it is shown that
\begin{equation}
 1 + \frac{1}{\varphi} = \varphi. 
\end{equation}
Multiplying by $\varphi$ gives
\begin{equation*}
\varphi + 1 = \varphi^2
\end{equation*}
which can be rearranged to
\begin{equation*}
{\varphi}^2 - \varphi - 1 = 0.
\end{equation*}
Using the quadratic formula, two solutions are obtained:
\begin{equation*}
\varphi = \frac{1 + \sqrt{5}}{2} = 1.61803\,39887\dots
\end{equation*}
and
\begin{equation*}
\varphi = \frac{1 - \sqrt{5}}{2} = -0.6180\,339887\dots
\end{equation*}
Because $\varphi$ is the ratio between positive quantities $\varphi$ is necessarily positive:
\begin{equation*}
\varphi = \frac{1 + \sqrt{5}}{2} = 1.61803\,39887\dots .
\end{equation*}

Different representations of the golden ratio are given in Table~\ref{tab:goldenratio}.

\begin{table}[hbt!]\centering
\caption{Number representations of the golden ratio}
\label{tab:goldenratio}

\begin{tabular}{|l|l|}
\hline
Form & Representation\\\hline
Binary & 1.1001111000110111011\ldots\\\hline
Decimal & 1.6180339887498948482\ldots\\\hline
Hexadecimal	& 1.9E3779B97F4A7C15F39\ldots\\\hline
Continued fraction & $1 + \cfrac{1}{1 + \cfrac{1}{1 + \cfrac{1}{1 + \cfrac{1}{1 + \ddots}}}}$\\[6ex]\hline
Algebraic form & $\displaystyle\frac{1 + \sqrt{5}}{2}$\\[2ex]\hline
Infinite series & $\displaystyle\frac{13}{8}+\sum_{n=0}^{\infty}\frac{(-1)^{(n+1)}(2n+1)!}{(n+2)!\,n!\,4^{(2n+3)}}$\\[2ex]\hline
\end{tabular}
\end{table}